\documentclass[a4paper, 12pt, finnish]{article}
\usepackage{babel}
\usepackage{afterpage}
\usepackage[utf8]{inputenc}
\usepackage[T1]{fontenc}
\usepackage{amsmath}
\usepackage{geometry}
\geometry{a4paper}
\usepackage{graphicx}
\usepackage{float}
\usepackage{wrapfig}
\usepackage{caption}
\usepackage{eurosym}
\usepackage[section]{placeins}
\usepackage{url}
\usepackage[hidelinks]{hyperref}
\usepackage{hyperref}
\usepackage{subcaption}
\usepackage{lipsum}
\usepackage{changepage}
\usepackage[table,xcdraw]{xcolor}
\usepackage[export]{adjustbox}
\usepackage[backend=bibtex]{biblatex}
\def\UrlBreaks{\do\/\do-} %% Line breakit urleissa
\graphicspath{{./figures/}} % Kuvien lähde

\begin{document}

\thispagestyle{empty} % Ei näytä sivunumeroa tällä sivul6la
\section{Tiivistelmä}

\section{Mikä on MX Linux 18?}
\subsection{Miksi valita GNU/Linux-jakelu?}

\section{Jakelun asentaminen}
\subsection{Asentamisen vaihe 1}
Alaotsikot tulevat olemaan asennusohjelman vaiheita tarkemmin kuvaaviksi.
\subsection{Asentamisen vaihe 2}
Esimerkiksi: Sijaintiasetusten määrittäminen, osiointi yms...
\subsection{Asentamisen vaihe 3}
\subsection{Asentamisen vaihe ...}

\section{Käyttöjärjestelmän käyttö}

\subsection{Työpöytäympäristö}

\subsubsection{Ohjelmien käynnistäminen}

\subsubsection{Tiedostonhallinta}

\subsection{Käyttäjien hallinta}

\subsubsection{Kuinka luoda uusi käyttäjä?}

\subsubsection{Kuinka poistaa käyttäjä?}

\subsubsection{Kuinka antaa käyttäjälle ylläpito-oikeudet?}

\subsection{Kuinka kustomoida ulkoasua?}

\subsubsection{Tausta- ja lukitusnäyttökuvan vaihtaminen}

\subsubsection{Teeman vaihtaminen}

\subsubsection{Muut kustomointimahdollisuudet}


\end{document}
