\documentclass[a4paper, 12pt, finnish]{article}
\usepackage{babel}
\usepackage{afterpage}
\usepackage[utf8]{inputenc}
\usepackage[T1]{fontenc}
\usepackage{amsmath}
\usepackage{geometry}
\geometry{a4paper}
\usepackage{graphicx}
\usepackage{float}
\usepackage{wrapfig}
\usepackage{caption}
\usepackage{eurosym}
\usepackage[section]{placeins}
\usepackage{url}
\usepackage[hidelinks]{hyperref}
\usepackage{hyperref}
\usepackage{subcaption}
\usepackage{lipsum}
\usepackage{changepage}
\usepackage[table,xcdraw]{xcolor}
\usepackage[export]{adjustbox}
\usepackage[backend=bibtex]{biblatex}
\def\UrlBreaks{\do\/\do-} %% Line breakit urleissa
\graphicspath{{./figures/}} % Kuvien lähde

\title{MX Linux 18 käyttöönotto, käyttö ja ohjeistus \\ \large Loppuraportti} %% \large == alaotsikko
\author{Iiro Aarnio \\ Tampereen seudun ammattiopisto \\}

\begin{document}

%% Kansisivu
%% Lisää pvm mm.
\maketitle
\thispagestyle{empty} % Ei näytä sivunumeroa tällä sivulla

\newpage
\thispagestyle{empty}
\begin{table}[htpb]
	\begin{tabular}{llll}
		Versiohistoria &            &                         &             \\
		\rowcolor[HTML]{FFCCC9}
		Versio         & Päivämäärä & Muutosperuste           & Tekijä      \\
		1.0              & 3.8.2019   & Dokumentti valmis       & Iiro Aarnio \\
	\end{tabular}
\end{table}


\newpage
\thispagestyle{empty} % Ei näytä sivunumeroa tällä sivul6la

\tableofcontents

\newpage
\pagenumbering{arabic} %% resettaa sivunro lask

\setcounter{page}{1} % Sisällysluettelo aloittaa laskennan tältä sivulta
\newpage
\section{Taustaa}

\begin{adjustwidth}{0.25\textwidth}{}
    Projektin tehtävänä oli luoda käyttöopas GNU/Linux-jakelusta ja harjoitella samalla projektityöskentelyä.
\end{adjustwidth}
\section{Saavutetut tulokset}

\begin{adjustwidth}{0.25\textwidth}{}
    Käyttöopas ja muut tarvittavat dokumentit saatiin tehtyä. Opas sisältää suunnitellut asiat.
\end{adjustwidth}

\section{Työn eteneminen}

\begin{adjustwidth}{0.25\textwidth}{}
    Käyttöoppaan suunnittelua varten tehtiin sisällysluettelosuunnitelmaa, joka oli suuntaa-antava.
    Lisäksi työaika aikataulutettiin. Kyseessäolevaa jakelua alettiin opiskella, jonka jälkeen alettiin rakentaa opasta palasista.

\end{adjustwidth}

\section{Kustannukset}

\begin{adjustwidth}{0.25\textwidth}{}
    Projektista ei tullut kustannuksia, kuten suunniteltu.
\end{adjustwidth}

\section{Resurssien käyttö}

\begin{table}[htpb]
\centering
\label{tab:tyo}
\begin{tabular}{lll}
\textbf{Resurssi}                & \textbf{Suunniteltu työmäärä} & \textbf{Toteutunut työmäärä} \\ \hline
\multicolumn{1}{l|}{Iiro Aarnio} & 120,5                         & 47
\end{tabular}
\end{table}

\section{Kokemukset}

\begin{adjustwidth}{0.25\textwidth}{}
    Aikataulutus oli turhan löysä. Paremmalla aikataulukuksella olisi voinut saada paremman lopputuloksen. Lopputulokseen kuitenkin ollaan tyytyväisiä ja työskentely on ollut opettavaista.
\end{adjustwidth}
\end{document}
