\documentclass[a4paper, 12pt, finnish]{article}
\usepackage{babel}
\usepackage{afterpage}
\usepackage[utf8]{inputenc}
\usepackage[T1]{fontenc}
\usepackage{amsmath}
\usepackage{geometry}
\geometry{a4paper}
\usepackage{graphicx}
\usepackage{float}
\usepackage{wrapfig}
\usepackage{caption}
\usepackage{eurosym}
\usepackage[section]{placeins}
\usepackage{url}
\usepackage[hidelinks]{hyperref}
\usepackage{hyperref}
\usepackage{subcaption}
\usepackage{lipsum}
\usepackage{changepage}
\usepackage[table,xcdraw]{xcolor}
\usepackage[export]{adjustbox}
\usepackage[backend=bibtex]{biblatex}
\def\UrlBreaks{\do\/\do-} %% Line breakit urleissa
\graphicspath{{./figures/}} % Kuvien lähde

\title{MX Linux 18 käyttöönotto, käyttö ja ohjeistus \\ \large Pohja} %% \large == alaotsikko
\author{Iiro Aarnio \\ Tampereen seudun ammattiopisto \\}

\begin{document}

%% Kansisivu
%% Lisää pvm mm.
\maketitle
\thispagestyle{empty} % Ei näytä sivunumeroa tällä sivulla

\newpage
\thispagestyle{empty}
\begin{table}[htpb]
	\begin{tabular}{llll}
		Versiohistoria &            &                         &             \\
		\rowcolor[HTML]{FFCCC9}
		Versio         & Päivämäärä & Muutosperuste           & Tekijä      \\
		0.1              & 16.4.2019   & Ensimmäinen vedos       & Iiro Aarnio \\
	\end{tabular}
\end{table}

\begin{table}[htpb]
	\begin{tabular}{lll}
		Jakelu &            &                                  \\
		\rowcolor[HTML]{FFCCC9}
		Tekijä         & Tulostettu & Jakelu                 \\
		Iiro Aarnio              & 16.4.2019   & Ei jakelua \\
	\end{tabular}
\end{table}

\newpage
\thispagestyle{empty} % Ei näytä sivunumeroa tällä sivul6la

\tableofcontents

\newpage
\pagenumbering{arabic} %% resettaa sivunro lask

\setcounter{page}{1} % Sisällysluettelo aloittaa laskennan tältä sivulta
\newpage
\section{<++>}

\subsection{<++>}

Normaali tapahtumien kulku

\begin{adjustwidth}{0.4\textwidth}{}
        <++>
\end{adjustwidth}
Vaihtoehtoinen tapahtumien kulku

\begin{adjustwidth}{0.4\textwidth}{}
        <++>
\end{adjustwidth}

\end{document}
