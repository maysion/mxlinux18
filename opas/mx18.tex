\documentclass[a4paper, 12pt, finnish]{article}
\usepackage{babel}
\usepackage{afterpage}
\usepackage[utf8]{inputenc}
\usepackage[T1]{fontenc}
\usepackage{amsmath}
\usepackage[margin=0.9in]{geometry}
\geometry{a4paper}
\usepackage{graphicx}
\usepackage{float}
\usepackage{wrapfig}
\usepackage{caption}
\usepackage{eurosym}
\usepackage[section]{placeins}
\usepackage{url}
\usepackage[hidelinks]{hyperref}
\usepackage{hyperref}
\usepackage{subcaption}
\usepackage{lipsum}
\usepackage{changepage}
\usepackage{bookmark}
\usepackage[table,xcdraw]{xcolor}
\usepackage[export]{adjustbox}
\definecolor{grey}{rgb}{0.9,0.9,0.9}
\def\UrlBreaks{\do\/\do-}
\graphicspath{{./figures/}} 
\begin{document}
\begin{titlepage}
	\colorbox{grey}{
		\parbox[t]{0.93\textwidth}{
			\parbox[t]{0.91\textwidth}{ 
				\raggedleft 
				\fontsize{80pt}{80pt}\selectfont 
				\vspace{0.7cm} 
				\textbf{MX Linux 18\\
				käyttöopas\\}
				\vspace{0.7cm}  
			}
		}
	}
	\vfill
	\parbox[t]{0.93\textwidth}{ 
		\raggedleft 
		\large 
		{\Large Iiro Aarnio}\\[4pt] 
		Projektissa toimiminen\\
		Tampereen seudun ammattiopisto\\[4pt] 
		\url{github.com/maysion/mxlinux18}\\
		\hfill\rule{0.2\linewidth}{1pt}
	}
\end{titlepage}
\thispagestyle{empty}
\begin{abstract}
	Tässä käyttöoppaassa perehdytään MX Linux 18 -GNU/Linux-jakelun käyttöönottoon ja käyttöön. Käyttöopas on suunnattu henkilöille, joilla ei ole aiempaa kokemusta Linux-järjestelmistä. Käyttöopas sisältää muun muassa käyttöjärjestelmän asentamisen, ohjelmien hakemisen ja päivittämisen. Opas ei sisällä laitteiston valmistelua.
\end{abstract}

\newpage
\thispagestyle{empty} 
\tableofcontents
\newpage
\pagenumbering{arabic} 
\setcounter{page}{1} 
\newpage

%%% DOC

\section{}


\end{document}
