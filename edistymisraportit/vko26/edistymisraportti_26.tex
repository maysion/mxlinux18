\documentclass[a4paper, 12pt, finnish]{article}
\usepackage{babel}
\usepackage{afterpage}
\usepackage[utf8]{inputenc}
\usepackage[T1]{fontenc}
\usepackage{amsmath}
\usepackage{geometry}
\geometry{a4paper}
\usepackage{graphicx}
\usepackage{float}
\usepackage{wrapfig}
\usepackage{caption}
\usepackage{eurosym}
\usepackage[section]{placeins}
\usepackage{url}
\usepackage[hidelinks]{hyperref}
\usepackage{hyperref}
\usepackage{subcaption}
\usepackage{lipsum}
\usepackage{changepage}
\usepackage[table,xcdraw]{xcolor}
\usepackage[export]{adjustbox}
\usepackage[backend=bibtex]{biblatex}
\def\UrlBreaks{\do\/\do-} %% Line breakit urleissa
\graphicspath{{./figures/}} % Kuvien lähde

\title{MX Linux 18 käyttöönotto, käyttö ja ohjeistus \\ \large Projektin edistymisraportti} %% \large == alaotsikko
\author{Iiro Aarnio \\ Tampereen seudun ammattiopisto \date{28. kesäkuuta 2019}}

\begin{document}

%% Kansisivu
%% Lisää pvm mm.
\maketitle
\thispagestyle{empty} % Ei näytä sivunumeroa tällä sivulla

\newpage
\thispagestyle{empty}
\begin{table}[htpb]
	\begin{tabular}{llll}
		Versiohistoria &            &                         &             \\
		\rowcolor[HTML]{FFCCC9}
		Versio         & Päivämäärä & Muutosperuste           & Tekijä      \\
		1.0              & 28.6.2019   & Dokumentti valmis       & Iiro Aarnio \\
	\end{tabular}
\end{table}

%\begin{table}[htpb]
	%\begin{tabular}{lll}
		%Jakelu &            &                                  \\
		%\rowcolor[HTML]{FFCCC9}
		%Tekijä         & Tulostettu & Jakelu                 \\
		%Iiro Aarnio              & 16.4.2019   & Ei jakelua \\
	%\end{tabular}
%\end{table}

\newpage
\thispagestyle{empty} % Ei näytä sivunumeroa tällä sivul6la

\tableofcontents

\newpage
\pagenumbering{arabic} %% resettaa sivunro lask

\setcounter{page}{1} % Sisällysluettelo aloittaa laskennan tältä sivulta
\newpage
\section{Aikataulutilanne}
Sairaustapausten vuoksi projekti on aikataulusta hieman jäljessä. Tämän hetkisen arvion mukaan, projekti saadaan vedettyä loppuun asti. Työhön on jouduttu käyttämään vähemmän tunteja, kuin suunniteltu. Loppuvaiheelle arvioidaan jäävän hyvin aikaa tehdä tarvittavia loppukorjailuja.

\section{Käytetyt resurssit}

\subsection{Kumulatiivinen ajankäyttö}
Suunnitteluun käytetyt tunnit.
\begin{table}[htpb]
\begin{tabular}{|l|l|}
\hline
Resurssi & Käytetyt tunnit \\ \hline
Iiro Aarnio & 10,5 \\ \hline
\end{tabular}%
\end{table}

Työhön käytetyt tunnit.
\begin{table}[htpb]
\begin{tabular}{|l|l|}
\hline
Resurssi & Käytetyt tunnit \\ \hline
Iiro Aarnio & 30 \\ \hline
\end{tabular}%
\end{table}

\subsection{Ajankäyttö osatehtäviin}

Suunnitteluvaiheen osatehtäviin käytetyt tunnit.
\begin{table}[!htpb]
\begin{tabular}{|l|l|}
\hline
Osatehtävä & Käytetyt tunnit \\ \hline
Projektiesityksen laatiminen & 1 \\ \hline
Projektisuunnitelman laatiminen & 3 \\ \hline
Palaverimuistioiden laatiminen & 1 \\ \hline
Aikataulun suunnittelu & 3,5 \\ \hline
Sisällysluettelon suunnittelu & 2 \\ \hline
    \textbf{Yhteensä} & \textbf{10,5}  \\ \hline
\end{tabular}%
\end{table}

Työskentelyvaiheen osatehtäviin käytetyt tunnit.
\begin{table}[!htpb]
\begin{tabular}{|l|l|}
\hline
Osatehtävä & Käytetyt tunnit \\ \hline
Jakelun opiskelu & 5 \\ \hline
Aloitus, kansisivu, tiivistelmä yms & 2 \\\hline
Asennuksesta kirjoittaminen & 10 \\\hline
Peruskäytöstä kirjoittaminen & 13 \\\hline
    \textbf{Yhteensä} & \textbf{30}  \\ \hline
\end{tabular}%
\end{table}

\section{Kustannukset}

Projektista ei ole koitunut kustannuksia.

\section{Arvio projektin kestosta}

Arvion mukaan projekti saadaan valmiiksi aikataulun mukaisesti.

\section{Ehdotus jatkotoimenpiteiksi}

Projektia jatketaan normaalisti.
\end{document}
